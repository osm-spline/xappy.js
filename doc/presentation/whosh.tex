\begin{frame} {whosh - a new approach to osm data import}
    \begin{block}{What?}
        \begin{itemize}
        \item tool to import osm geo data into a postgres database
        \item push data into the database as fast as possible
        \item actions that are not possible while streaming data are handled by the db
        \end{itemize}
    \end{block}
    \begin{block}{Why?}
        \begin{itemize}
        \item all available tools need a lot of time for the import
        \item all available tools need a lot of ram
        \item most available tools do more than we need
        \end{itemize}
    \end{block}
\end{frame}

\begin{frame} {whosh - a new approach to osm data import}
    \begin{block}{How?}
        \begin{itemize}
        \item we use osmium - a library for parsing osm data in multiple formats
        \item library executes callbacks on node, way or relation
        \item independent COPY command for nodes, way and relations
        \item COPY allows to insert data fast (and unchecked) into tables
        \item functions in pg/SQL create geometry objects
        \item streaming and using pg/SQL should solve ram and time issues
        \end{itemize}
    \end{block}    
\end{frame}

\begin{frame} {whosh - a new approach to osm data import}
    \begin{block}{Future?}
        \begin{itemize}
        \item currently text format for COPY \\
        $\rightarrow$ binary format would be better
        \item speed and usability benchmarks
        \item world domination ... eh import
        \end{itemize}
    \end{block}    
\end{frame}
